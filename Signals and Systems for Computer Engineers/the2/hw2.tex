\documentclass[10pt,a4paper, margin=1in]{article}
\usepackage{fullpage}
\usepackage{amsfonts, amsmath, pifont}
\usepackage{amsthm}
\usepackage{graphicx}
\usepackage{float}

\usepackage{tkz-euclide}
\usepackage{tikz}
\usepackage{pgfplots}
\pgfplotsset{compat=1.13}

\usepackage{geometry}
 \geometry{
 a4paper,
 total={210mm,297mm},
 left=10mm,
 right=10mm,
 top=10mm,
 bottom=10mm,
 }
 % Write both of your names here. Fill exxxxxxx with your ceng mail address.
 \author{
  Şamlıoğlu, Erdem\\
  \texttt{e2448843@ceng.metu.edu.tr}
  \and
  Uludoğan, Mert\\
  \texttt{e2380996@ceng.metu.edu.tr}
}

\title{CENG 384 - Signals and Systems for Computer Engineers \\
Spring 2023 \\
Homework 2}
\begin{document}
\maketitle



\noindent\rule{19cm}{1.2pt}

\begin{enumerate}

\item %write the solution of q1
    \begin{enumerate}
    % Write your solutions in the following items.
    \item $y^{'}(t)=x(t) - 5y(t)$
    %write the solution of q1a
    \item Our input is $x(t)=(e^{-t}+e^{-3t})u(t)$ and the system is initially at rest. To find the output $y(t)$ we can use the formula $y(t)= y_h(t) + y_p(t)$. For the homogenous part, we can use the formula: $y_h(t)=Ke^{a}t$ From the part a, $y^{'}+5y(t)=0$, we can derive $(a+5)Ke^{a}t=0$. So $a=-5$, and our $y_h(t)=Ce^{-5t}$. \newline
    For the particular part, $y_p(t)=Ae^{-t}+ Be^{-3t}$. From part a we know, $y^{'}(t)+5y(t)=x(t)$. So we can calculate $y_p^{'}$ which is equal to $-Ae^{-t}+ -3Be^{-3t}$. Putting the values we have found in the formula, we get: \newline
    $-Ae^{-t}+ -3Be^{-3t} + 5Ae^{-t}+ 5Be^{-3t}=e^{-t}+ e^{-3t}$. simplifying the equation:\newline
    $4Ae^{-t}+ 2Be^{-3t}=e^{-t}+ e^{-3t}$. We can see that $4A=1$ and $2B=1$. So accordingly; $A=\frac{1}{4}$, $B=\frac{1}{2}$. Now we can put the values and find: \newline
    $y_p(t)=\frac{1}{4}e^{-t}+\frac{1}{2}e^{-3t}$. so $y(t)=Ce^{-5t} + \frac{1}{4}e^{-t}+\frac{1}{2}e^{-3t}$. To find the value C, we know that the system is initially at rest, which means $y(0)=y^{'}(0)= y^{''}(0)=...=0$ For t=0, $y(0)=C+\frac{1}{4} + \frac{1}{2}=0 $. So C is equal to $\frac{-3}{4}$. Putting C in the equation we find finally: \newline
    $y(t)=\frac{-3}{4}e^{-5t} + \frac{1}{4}e^{-t}+\frac{1}{2}e^{-3t}$
    
    
    
    
    
    
    %write the solution of q1b
    \end{enumerate}

\item %write the solution of q2  
	\begin{enumerate}
    % Write your solutions in the following items.
    \item %write the solution of q2a
        \text{}\linebreak
        \text{$x[n] = 2\delta[n] + \delta[n+1]$}\vspace{2pt}\linebreak
        \text{Thanks to the "distributivity" property of Convolution, we can seperate the impulse function}\linebreak
        \text{to use one-term impulse response functions.}\vspace{2pt}\linebreak
        \text{$h[n] = \delta[n-1]+2\delta[n+1]$}\vspace{2pt}\linebreak
        \text{$h_{0}[n] = \delta[n-1]$, $h_{1}[n] = 2\delta[n+1]$}\vspace{2pt}\linebreak
        \text{for $h_{0}[n]$:}\vspace{2pt}\linebreak
        \text{$y_{0}[n] = x[n] \ast h_{0}[n] = 2\delta[n-1] + \delta[n]$, $h_{0}$ is clearly shift the current input}\vspace{2pt}\linebreak
        \text{for $h_{1}[n]$:}\vspace{2pt}\linebreak
        \text{$y_{1}[n] = x[n] \ast h_{1}[n] = 4\delta[n+1]+2\delta[n+2]$}\vspace{2pt}\linebreak
        \text{$y[n] = y_{0}[n] + y_{1}[n]$}\vspace{2pt}\linebreak
        \text{$= 2\delta[n-1] + \delta[n] + 4\delta[n+1]+2\delta[n+2]$}\vspace{2pt}\linebreak

       

         
    \item %write the solution of q2b
        \text{}\linebreak
        \text{Firstly, let's take the derivative of $x(t)$, where $x(t) = u(t-1) + u(t+1)$}\vspace{2pt}\linebreak
        \text{$\frac{d x(t)}{t} = \delta(t-1) + \delta(t+1)$}\vspace{2pt}\linebreak
        \text{$h(t) = e^{-t}\cdot sin(t) \cdot u(t)$}\vspace{2pt}\linebreak
        \text{Using "commutative" property, we can change the position of $\frac{d x(t)}{t}$ and h(t) in convolution function:}\vspace{2pt}\linebreak
        \text{Let's say, $w(t) = e^{-t}\cdot sin(t)\cdot u(t)$ and $h(t) =  \delta(t-1) + \delta(t+1)$}\vspace{2pt}\linebreak
        \text{Now, using "distributivity" property, $h_{0}(t) = \delta(t-1)$ and $h_{1}(t) = \delta(t+1)$}\vspace{2pt}\linebreak
        \text{$y_{0}(t) = w(t) \ast h_{0}(t) = e^{-t+1}\cdot sin(t-1)\cdot u(t-1)$}\vspace{2pt}\linebreak
        \text{$y_{1}(t) = w(t) \ast h_{1}(t) = e^{-t-1}\cdot sin(t+1)\cdot u(t+1)$}\vspace{2pt}\linebreak
        \text{$y(t) = y_{0} + y_{1}$}\vspace{2pt}\linebreak
    \end{enumerate}

\item %write the solution of q3
    \begin{enumerate}
    % Write your solutions in the following items.
    \item To find $y(t)= x(t) * h(t)$, we calculate: $\int_{\infty}^{-\infty} x(\tau)h(t-\tau)d\tau$. For the given values in part a, we get : $\int_{0}^{t} e^{-\tau} e^{-2t + 2\tau}d\tau$ which is equal to $\int_{0}^{t} e^{-2t + \tau}d\tau$. Taking the constant $e^{-2t}$ out, we get $e^{-2t}\int_{0}^{t} e^{\tau}d\tau$, Using the common integral and computing the boundaries, finally we get: $e^{-2t}(1-e^{t})u(t)$
    %write the solution of q3a
    \item To find $y(t)= x(t) * h(t)$ we need to calculate different regions for x(t) because it behaves differently for different values of t. \newline
    For $0<t<1$, we need to calculate: $\int_{0}^{t} e^{3t - 3\tau}d\tau$. Taking the $e^{3t}$ constant out, $e^{3t}\int_{0}^{t} e^{-3\tau}d\tau$. Applying u substitution and using common integral we get: $\frac{-1}{3}e^{3t}(e^{-3t}-1)(u(t)-u(t-1))$ \newline
    For $t>1$, we need to calculate $\int_{0}^{1} e^{3t - 3\tau}d\tau$. Taking the constant $e^{3t}$ constant out, $e^{3t}\int_{0}^{1} e^{-3\tau}d\tau$. Applying u substitution and using common integral we get: $\frac{(e^{3}-1)e^{3t-3}}{3} u(t-1) $
    
    
    
    %write the solution of q3b
    \end{enumerate}

\item %write the solution of q4
    \begin{enumerate}   
    % Write your solutions in the following items.
    \item %write the solution of q4a
        \text{}\linebreak
        \text{$y[n] - y[n-1] - y[n-2] = 0$, $y[0] = 1$ and $y[1] = 1$}\vspace{2pt}\linebreak
        \text{Firstly, we can write characteristic equation for this recurrence relation,}\vspace{2pt}\linebreak
        \text{which is $r^{2} - r - 1 = 0$}\vspace{2pt}\linebreak
        \text{$r_{1} = \dfrac{(1+\sqrt{5})}{2}$ and $r_{2} = \dfrac{(1-\sqrt{5})}{2}$}\vspace{2pt}\linebreak
        \text{so the general solution: $y[n] = A\cdot r_{1}^{n} + B\cdot r_{2}^{n}$}\vspace{2pt}\linebreak
        \text{A and B can be found from the initial conditions.}\vspace{2pt}\linebreak
        \text{$y[0] = 1$ implies $A + B = 1$}\vspace{2pt}\linebreak
        \text{$y[1] = 1$ implies $r_{1}\cdot A + r_{2}\cdot B = 1$}\vspace{2pt}\linebreak
        \text{as results of calculations:}\vspace{2pt}\linebreak
        \text{$A = \dfrac{(1+\sqrt{5})}{2\sqrt{5}}$}\vspace{2pt}\linebreak
        \text{$B = \dfrac{(1-\sqrt{5})}{2\sqrt{5}}$}\vspace{2pt}\linebreak
        \text{$y[n] = \dfrac{1}{\sqrt{5}}\cdot \Big(\big(\dfrac{1+\sqrt{5}}{2}\big)^{n} + \big(\dfrac{1-\sqrt{5}}{2}\big)^{n}\Big)$}\vspace{2pt}\linebreak
        \text{$$}
    \item %write the solution of q4b
        \text{}\linebreak
        \text{$y^(3)(t)-6y^{(2)}+13y^{(1)}-10y+0$}\vspace{2pt}\linebreak
        \text{This is an homogenous differential }\vspace{2pt}\linebreak
        \text{$(\alpha^{3}-\alpha^{2}+13\alpha-10)\cdot C\cdot e^{\alpha t} = 0$}\vspace{2pt}\linebreak
        \text{$(\alpha^{3}-\alpha^{2}+13\alpha-10) = 0$}\vspace{2pt}\linebreak
        \text{$r_{1} = 2, r_{2} = (2+i), r_{3} = (2-i)$}\vspace{2pt}\linebreak
        \text{$y = C_{1}e^{2t} + C_{2}e^{2t}cost + C_{3}e^{2t}sint$}\vspace{2pt}\linebreak
        \text{$y(0) = 1, y^{(1)}(0) = \dfrac{3}{2}, y^{(2)}(0) = 3$}\vspace{2pt}\linebreak
        \text{As a result of computations:}\vspace{2pt}\linebreak
        \text{$y=2e^2t-e^{2t}cost-\dfrac{1}{2}e^{2t}sint$}
    \end{enumerate}

\item %write the solution of q5
    \begin{enumerate}
    % Write your solutions in the following items.
    \item To find the particular solution for $x(t)=cos(5t)$, guess a solution: $y(t)= Acos(5t)+Bsin(5t)$. Now we should find the first and second derivatives accordingly: \newline
    $y^{'}(t)= -5Asin(5t)+5Bcos(5t)$\newline
    $y^{''}(t)= -25Acos(5t)-25Bsin(5t)$\newline
    Putting what we have found in the equation: $(-25Acos(5t)-25Bsin(5t)) + 5(-5Asin(5t)+5Bcos(5t)) + 6(Acos(5t)+Bsin(5t)) =cos(5t) $ \newline
    Now we can find $-25A+25B+6A=1$ and $-25B-25A+6B=0$. $A=\frac{-19}{986}$ . $B=\frac{25}{986}$   \newline
    $y(t)= \frac{-19}{986}cos(5t)+ \frac{25}{986}sin(5t)$
    
    
     
    
    %write the solution of q5a
    \item To find the homogeneous solution, we can use the characteristic equation which is: \newline
    $r^{2}+5r+6$ for this function. We can see from this equation, roots are $r_1=-2$, and $r_2=-3$. \newline
    So accordingly we can find $y_h(t)=Ce^{-2t} + De^{-3t}$ for some constants C,D.
    
    
    %write the solution of q5b
	\item For general solution, we add up part a and b, we get : \newline
    $y(t)= Ce^{-2t} + De^{-3t} + \frac{-19}{986}cos(5t)+ \frac{25}{986}sin(5t)$. and we also find it's derivative: \newline
    $y^{'}(t)= (-2Ce^{-2t}  -3De^{-3t}) + (-5\frac{-19}{986}sin(5t)+ 5\frac{25}{986}cos(5t))$ Since it's initially at rest. We can say that $y(0)=0$ and $y^{'}(0)=0$. So putting the values, $y(0)= C + D + \frac{-19}{986} =0 $, $ C+ D =\frac{19}{986}$. \newline
    $y^{'}(0)= -2C -3D + \frac{125}{986}=0$ , $2C+3D=\frac{125}{986}$ \newline
    So we can conclude as, $C=\frac{-70}{986}$, $D=\frac{89}{986}$. applying it to the general solution we get: \newline
    $y(t)= \frac{-70}{986}e^{-2t} + \frac{89}{986}e^{-3t} + \frac{-19}{986}cos(5t)+ \frac{25}{986}sin(5t)$
    
    
 
 
 
 
 %write the solution of q5c
    \end{enumerate}    
    
\item %write the solution of q6
    \begin{enumerate}
    % Write your solutions in the following items.
    \item %write the solution of q6a
        \text{}\linebreak
        \text{$w[n] - \frac{1}{2}w[n-1] = x[n]$}\vspace{8pt}\linebreak
        \text{$2w[n] - w[n-1] = 2x[n]$}\vspace{2pt}\linebreak
        \text{$2w[n-1] - w[n-2] = 2x[n-1]$}\vspace{2pt}\linebreak
        \text{$2w[n-2] - w[n-3] = 2x[n-2]$}\vspace{2pt}\linebreak
        \text{...}\vspace{2pt}\linebreak
        \text{(Multiplying and summing to eliminate left hand side except $w[n]$)}\vspace{2pt}\linebreak
        \text{$2^{n}\cdot w[n] - w[0] = \sum_{k=0}^{n-1}x[n-k]\cdot 2^{n-k}$}\vspace{2pt}\linebreak
        \text{$w[n] - w[0] = \sum_{k=0}^{n-1}x[n-k]\cdot 2^{-k}$}\vspace{2pt}\linebreak
        \text{$w[0] = 0$ because the system is initially rest}\vspace{2pt}\linebreak
        \text{$h_{0}[n] = w[n] = \sum_{k=0}^{n-1}\delta[n-k]\cdot 2^{-k}$}\vspace{2pt}\linebreak
        \text{$h_{0}[n] = 2^{-n}u[n]$}\vspace{2pt}\linebreak

    \item %write the solution of q6b
        \text{}\linebreak
        \text{Overall impulse is the convolution of the serial models of the system}\vspace{2pt}\linebreak
        \text{$h[n] = h_{0}[n] \ast h_{0}[n]$}\vspace{4pt}\linebreak
        \text{$\sum_{k=-\infty}^{\infty}2^{-k}u[k]\cdot2^{k-n}u[n-k]$}\vspace{4pt}\linebreak
        \text{$\sum_{k=0}^{\infty}2^{-k}\cdot2^{k-n}u[n-k]$}\vspace{2pt}\linebreak
        \text{$\sum_{k=0}^{\infty}2^{-n}u[n-k]$}\vspace{2pt}\linebreak

	\item %write the solution of q6c
    \end{enumerate}
    
\item %write the solution of q7
    \begin{enumerate}
        \begin{verbatim}

        import numpy as np
        import matplotlib.pyplot as plt
        
        
        input_data = np.genfromtxt('hw2_signal.csv', delimiter=',')
        start_index = int(input_data[0])
        x = input_data[1:]
        
        
        h = np.zeros_like(x)
        h[start_index - 5] = 1
        
        # Perform discrete convolution of x[n] and h[n]
        y = np.zeros_like(x)
        for n in range(len(x)):
            for k in range(len(h)):
                if n - k >= 0:
                    y[n] += x[k] * h[n - k]
        
        
        n = np.arange(len(x))
        plt.stem(n, x, linefmt='b-', markerfmt='bo', label='Input Signal x[n]')
        plt.stem(n, y, linefmt='r-', markerfmt='ro', label='Output Signal y[n]')
        plt.xlabel('n')
        plt.ylabel('Amplitude')
        plt.legend()
        plt.show()

        \end{verbatim}


    
    
    
    %write the solution of q7a
    \item %write the solution of q7b
        \begin{verbatim}
            


import matplotlib.pyplot as plt
import numpy as np

def convolution(x, h):

    M = len(x)
    N = len(h)

    x_padded = np.pad(x, (0, N - 1), mode='constant')
    h_padded = np.pad(h, (0, M - 1), mode='constant')

    # Initialize the output array
    y = np.zeros(M + N - 1)

    # Perform convolution
    for n in range(M + N - 1):
        for k in range(N):
            if n - k < 0 or n - k >= M:
                continue
            y[n] += x_padded[n - k] * h_padded[k]

    print(type(y))
    return y

def draw_seven_b(list, start_index, N):

    output_values = list
    # Create a time axis
    n = np.arange(start_index, start_index + output_values.size, 1)

    # Plot the signal
    plt.stem(n, output_values)
    plt.xlabel('Time (n)')
    plt.ylabel('Amplitude')
    plt.title('Discrete Signal')
    plt.suptitle(f"N = {N}")
    plt.show()



def seven_b():
    with open("hw2_signal.csv") as csv_file:
        x = csv_file.readline().strip()
        _x = x.split(',')
        almost_data = list(map(float, _x))
        start_index = int(almost_data[0])
        data_xn = almost_data[1:]

        # draw_seven_b(np.array(data_xn), start_index, 33)

        for N in [3,5,10,20]:
            start_index_output = 0 - (N - 1)
            data_hn = [1/N]*(N)
            # Define the input arrays
            graph = data_xn
            function = data_hn

            result = convolution(graph, function)

            draw_seven_b(result, start_index_output, N)

seven_b()

        \end{verbatim}


    \end{enumerate}    

\end{enumerate}


\end{document}

